Si verifica ora se sussistono delle differenze notevoli di winrate tra i due team di Dota2. I winrate per ciascuno dei due team sono riportati all'interno della seguente tabella~\ref{winrate_team_sintetico}. Rispetto al dataset originale si può notare che la differenza di winrate tra i due team risulta essere molto più marcata. Se si dovesse interpretare l'insieme delle partite generate come un insieme di partite in un determinato stato del gioco, tale stato di gioco sarebbe da considerare estremamente sbilanciato a favore del team 1. 
\begin{table}
\centering
\caption{Tabella riassuntiva del winrate di ciascuno dei due team per le partite ranked e unranked}
\label{winrate_team_sintetico}
\begin{tabular}{|r|r|r|}
\hline
\multicolumn{1}{|l|}{} & \multicolumn{1}{l|}{ranked} & \multicolumn{1}{l|}{unranked} \\ \hline
team 1  &  57.77\%  &   55.66\% \\ \hline
team 2  &  42.23\%  &   44.34\%  \\ \hline
\end{tabular}
\end{table}