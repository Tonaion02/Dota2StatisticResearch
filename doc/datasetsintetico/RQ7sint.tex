Infine si procede a rispondere alla research question 7 anche per il dataset sintetico, all'interno della quale si cercherà di determinare se l’elo per le partite ranked e unranked segue o meno una distribuzione nota e se questa distribuzione individuata è riconducibile dal campione alla popolazione. Come per il dataset originale il valore dell’elo di ogni partita del dataset sintetico è stato calcolato attraverso l’utilizzo della medesima formula utilizzata per il dataset originale. \\
All’interno delle seguenti figure () sono presenti i grafici delle distribuzioni dell'elo, rispettivamente per le partite ranked e unranked. Proprio come nel caso del dataset originale entrambi i grafici ci suggeriscono che l’elo abbia una distribuzione normale. Quindi per determinare ciò abbiamo utilizzato il test del chi-quadro per verificare se la distribuzione è effettivamente una normale o meno.
Nel caso delle partite ranked il valore di chi-quadro è 15.16329 mentre per le partite unranked è 9.958115, dato che in entrambi i casi il valore non rientra nell’intervallo di accettazione del test del chi-quadro(0.05063562, 7.377759) si può affermare che l’elo sia per le partite ranked che per quelle unranked non segue una distribuzione normale. \\
\begin{figure}[htbp]
\centering
\begin{multicols}{2}
\hspace*{-0.1\linewidth}
\includesvg[scale=0.7]{./FigureSintetico/distribuzione_elo_ranked_sintetico.svg}
\caption{Nella seguente figura è presente un grafico della distribuzione dell'elo calcolato all'interno delle partite ranked}
\label{distribuzione_elo_ranked_sintetico}
\hspace*{-0.1\linewidth}
\includesvg[scale=0.7]{./FigureSintetico/distribuzione_elo_unranked_sintetico.svg}
\caption{Nella seguente figura è presente un grafico della distribuzione dell'elo calcolato all'interno delle partite unranked}
\label{distribuzione_elo_unranked_sintetico}
\end{multicols}
\end{figure} 
Di conseguenza, anche se la forma della distribuzione non sembra seguire una distribuzione di Poisson, si è comunque deciso di applicare il test del chi-quadro per verificarlo. Nel caso delle partite ranked, il valore del chi-quadro è 397.6751, mentre per le partite unranked è 879.4748. Poiché questi valori non rientrano minimamente nell’intervallo di accettazione del test, si può confermare che le osservazioni fatte sulle distribuzioni dell’Elo sono corrette. \\
Si conclude quindi che entrambe le distribuzioni non seguono alcuna distribuzione nota, il che impedisce di proseguire lo studio e di determinare il valore medio dell’Elo della popolazione per il dataset sintetico. 
