Nella seguente research question 6 si cercherà di definire se esistono dei team che sono più convenienti dal punto di vista statistico per ottenere la vittoria. Anche in questo caso si è deciso di scartare tutte quante le configurazioni di team che sono presenti in meno del 10\% delle partite. In questo caso abbiamo un team in meno nella top rispetto alle classifiche del dataset originale. \\
In questo caso come è possibile osservare dai winrate e dalle configurazioni dei team che sono presenti all'interno delle tabelle~\ref{tabella_winrate_team_ranked_sintetico}~\ref{tabella_winrate_team_unranked_sintetico} sia nelle partite ranked che in quelle unranked i team presenti sono gli stessi anche se con delle differenti percentuali di vittoria. \\
Come nel caso del dataset originale, le configurazioni di team più vincenti sono coerenti rispetto a quelli che sono i risultati ottenuti nelle research question precedente, quindi durante l'analisi del winrate per attributi. Si era infatti concluso che convenisse sia nelle partite ranked che in quelle unranked avere almeno un Hero per ciascun attributo per massimizzare le possibilità di vittoria. \\
In modo simile al dataset originale, si può concludere che anche in questo caso esistono configurazioni di team più convenienti rispetto ad altre. L’unica differenza significativa è che, nel dataset sintetico, le configurazioni di team più vincenti presentano una percentuale di vittoria decisamente più elevata.
\begin{table}[htbp]
\centering
\caption{Nella seguente tabella sono riportati le 4 configurazioni di team che hanno il winrate più alto all'interno delle partite ranked. }
\label{tabella_winrate_team_ranked_sintetico}
\begin{tabular}{|r|r|r|r|r|}
\hline
\multicolumn{1}{|l|}{Strength} & \multicolumn{1}{l|}{Intelligence} & \multicolumn{1}{l|}{Agility} & \multicolumn{1}{l|}{Universal} & \multicolumn{1}{l|}{WinRate} \\ \hline
2 & 1 & 1 & 1 &   58  \% \\ \hline
1 & 1 & 2 & 1 &   57  \% \\ \hline
1 & 2 & 1 & 1 &   49.5\%  \\ \hline
1 & 1 & 1 & 2 &   48.4\%  \\ \hline
\end{tabular}
\end{table}
\begin{table}[htbp]
\centering
\caption{Nella seguente tabella sono riportati le 4 configurazioni di team che hanno il winrate più alto all'interno delle partite unranked.}
\label{tabella_winrate_team_unranked_sintetico}
\begin{tabular}{|r|r|r|r|r|}
\hline
\multicolumn{1}{|l|}{Strength} & \multicolumn{1}{l|}{Intelligence} & \multicolumn{1}{l|}{Agility} & \multicolumn{1}{l|}{Universal} & \multicolumn{1}{l|}{WinRate} \\ \hline
1 & 1 & 2 & 1 &   57.1 \% \\ \hline
1 & 2 & 1 & 1 &   52.7 \%  \\ \hline
2 & 1 & 1 & 1 &   50.2 \%  \\ \hline
1 & 1 & 1 & 2 &   43.5 \% \\ \hline
\end{tabular}
\end{table}