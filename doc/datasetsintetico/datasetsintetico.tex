\subsection{RQ1:Il winrate varia notevolmente tra gli Hero?}
Si cerca ora di rispondere alla research question numero 1 anche per il dataset sintetico.
\begin{figure}[htbp]
\begin{center}
\includesvg[scale=0.9]{./FigureSintetico/distribuzione_winrate.svg}
\caption{Barplot che mostra il winrate per ciascuno degli Hero}
\label{winrate_totale_sint}
\end{center}
\end{figure}
Nella figura~\ref{winrate_totale_sint} è riportato il barplot per il winrate di ciascuno degli Hero, anche all'interno del dataset sintetico è presente una certa variabilità del winrate. In particolare la varianza del winrate è pari a 0.005914222. Seppure non c'è una forte varianza quest'ultima risulta essere maggiore rispetto a quella del dataset non sintetico.
\begin{figure}[htbp]
\begin{center}
\includesvg[scale=0.85]{./FigureSintetico/frequenze_winrate_ranked.svg}
\caption{Barplot che mostra il winrate per ciascuno degli Hero nelle partite ranked}
\label{winrate_ranked_sintetico}
\includesvg[scale=0.85]{./FigureSintetico/frequenze_winrate_unranked.svg}
\caption{Barplot che mostra il winrate per ciascuno degli Hero nelle partite unranked}
\label{winrate_unranked_sintetico}
\end{center}
\end{figure}
Anche per il dataset sintetico è stato deciso di analizzare in maniera distinta le partite ranked da quelle unranked. La varianza degli hero nelle partite ranked è 0.00697383 e quella delle partite unranked è 0.006019371. Non c'è una particolare variazione neanche in questo caso. Non si notano notevoli differenze neanche in questo caso tra ranked e unranked andando a guardare i barplot nelle figure~\ref{winrate_ranked_sintetico}~\ref{winrate_unranked_sintetico}.
Anche per il dataset sintetico si sono analizzate le distribuzioni dei winrate per le partite ranked e quelle unranked(figura~\ref{distribuzione_winrate_ranked_unranked_sintetico}). In questo caso si può notare un notevole peggioramento nella distribuzione rispetto al winrate degli Hero rispetto al dataset originale. Difatti, si hanno ben il 47\% degli Hero per le ranked e il 57\% per le unranked che si trovano fuori dal range (0.45, 0.55). Di conseguenza, il gioco risulta avere addirittura una maggiore variabilità per quanto concerne il winrate degli Hero. 
\begin{figure}[htbp]
\begin{center}
\includesvg[scale=0.75]{./FigureSintetico/distribuzione_winrate_ranked_unranked_sintetico.svg}
\caption{Figura che mostra i barplot per la distribuzione del winrate nelle partite ranked e in quelle unranked}
\label{distribuzione_winrate_ranked_unranked_sintetico}
\end{center}
\end{figure}

\subsection{RQ2:winrate e pickrate variano tra ranked e unranked?}
Allo scopo di verificare se all'interno del dataset sintetico è presente una variabilità tra le partite classificate e quelle non classificate, si procede a rispondere alla research question 2 per il dataset sintetico. \\
Nelle figura~\ref{frequenze_pickrate_ranked_unranked_sintetico} sono riportati i pickrate degli Hero nelle partite ranked e unranked. La differenza dei pickrate tra le due modalità è osservabile meglio all'interno del grafico~\ref{differenza_pickrate_ranked_unranked_sintetico}. Rispetto al dataset originale, è possibile osservare che le differenze tra le modalità ranked e unranked sono molto minori. In questo caso praticamente non abbiamo mai un Hero che supera la soglia del 2.5\%.
Sono stati poi realizzati anche in questo caso i boxplot ad intaglio~\ref{boxplot_pickrate_sintetico} per il pickrate degli Hero per le partite ranked e unranked. Si nota la quasi totale assenza di outlier, che ritorna come informazione rispetto al grafico delle differenze dei pickrate~\ref{differenza_pickrate_ranked_unranked_sintetico}. Come nel caso del dataset originale, l'intervallo (-0.002648102, 0.001843585) suggerisce che non ci sono differenze statistiche particolari tra le mediane dei due campioni. \\
Nella tabella~\ref{indici_sintesi_pickrate_sintetico} sono stati riportati gli indici di sintesi per il pickrate nelle partite ranked e unranked. Se la media dei campioni è rimasta invariata rispetto al dataset originale, è interessante notare che la mediana è invece cambiata. La varianza e la deviazione standard sono diminuite moltissimo rispetto al campione originale. \\
I valori del campione sintetico si allontanano di più da quelli dell'originale proprio per skewness e curtosi. Quindi, a differenza del campione originale, nel campione sintetico sussistono delle differenze importanti riguardo la distribuzione del pickrate, nonostante non ci siano delle forti variazioni per quanti riguarda i singoli Hero. Infatti rispettivamente per le partite ranked e unranked abbiamo una skewness positiva e negativa, che comportano quindi rispettivamente un asimmetria a destra e una a sinistra. 
\begin{figure}[htbp]
\begin{center}
\hspace*{-0.15\linewidth}
\includesvg[scale=0.9]{./FigureSintetico/differenze_ranked_unranked/frequenze_pickrate_ranked_unranked_sintetico.svg}
\caption{Nella seguente figura sono mostrati i barplot per il pickrate degli hero nelle partite ranked e unranked}
\label{frequenze_pickrate_ranked_unranked_sintetico}
\end{center}
\end{figure}
\begin{figure}[htbp]
\begin{center}
\hspace*{-0.15\linewidth}
\includesvg[scale=1.1]{./FigureSintetico/differenze_ranked_unranked/differenza_pickrate_ranked_unranked_sintetico.svg}
\caption{Nella seguente figura è mostrato il barplot per la differenza del pickrate degli hero nelle partite ranked e unranked}
\label{differenza_pickrate_ranked_unranked_sintetico}
\end{center}
\end{figure}
\begin{figure}[htbp]
\begin{center}
% \hspace*{-0.15\linewidth}
\includesvg[scale=0.9]{./FigureSintetico/differenze_ranked_unranked/boxplot_new_sintetico.svg}
\caption{Nella seguente figura sono mostrati i boxplot ad intaglio per il pickrate nelle partite ranked e unranked}
\label{boxplot_pickrate_sintetico}
\end{center}
\end{figure}
\begin{table}
\centering
\caption{Tabella riassuntiva degli indici di sintesi per il pickrate per le partite ranked e unranked}
\label{indici_sintesi_pickrate_sintetico}
\begin{tabular}{|r|r|r|}
\hline
\multicolumn{1}{|l|}{} & \multicolumn{1}{l|}{ranked} & \multicolumn{1}{l|}{unranked} \\ \hline
media               &  0.08850  &   0.08850 \\ \hline
mediana             &  0.08820  &  0.08860 \\ \hline
deviazione standard &  0.005701508  &   0.005403042 \\ \hline
varianza            &  3.25072e-05  &   2.919287e-05 \\ \hline
\end{tabular}
\end{table}
\begin{table}
\centering
\caption{Tabella riassuntiva degli indici simmetrici per pickrate nelle partite ranked e unranked}
\label{indici_simmetrici_pickrate_sintetico}
\begin{tabular}{|r|r|r|}
\hline
\multicolumn{1}{|l|}{} & \multicolumn{1}{l|}{ranked} & \multicolumn{1}{l|}{unranked} \\ \hline
skewness    &  0.1577882  &   -0.3684828     \\ \hline
curtosi     &  2.848794  &   2.547602     \\ \hline
\end{tabular}
\end{table}
Si verifica anche per il winrate se ci sono delle differenze interessanti tra i due campioni analizzati. All'interno della figura~\ref{boxplot_winrate_sintetico} sono riportati i boxplot ad intaglio del winrate. Come nel caso del pickrate, anche per il winrate si hanno un basso numero di outlier e anche in questo caso l'intervallo di confidenza(-0.03541395, 0.02387649) indica che le mediane dei due campioni non sono troppo differenti tra di loro. \\
Gli indici di sintesi riportati all'interno della seguente tabella~\ref{indici_sintesi_winrate_sintetico} non mostrano particolari differenze tra media, mediana e varianza tra le partite ranked e unranked. Rispetto al dataset originale è possibile notare un minimo incremento degli indici di dispersione. \\
In questo caso le distribuzioni non mutano sensibilmente tra le partite ranked e quelle unranked, come è possibile vedere nella figura~\ref{distribuzione_winrate_ranked_unranked_sintetico}, il che è testimoniato anche dagli indici presenti nella tabella~\ref{indici_simmetrici_winrate_sintetico}. Rispetto al dataset originale però, si hanno meno differenze tra gli Hero che superano quella che è la soglia stabilita. In particolare 27.4\% contro il 37\% del dataset originale. \\
\begin{figure}[htbp]
\begin{center}
% \hspace*{-0.15\linewidth}
\includesvg[scale=0.9]{./FigureSintetico/differenze_ranked_unranked/boxplot_winrate_sintetico.svg}
\caption{Nella seguente figura sono mostrati i boxplot ad intaglio per il winrate nelle partite ranked e unranked}
\label{boxplot_winrate_sintetico}
\end{center}
\end{figure}
\begin{table}
\centering
\caption{Tabella riassuntiva degli indici di sintesi per il winrate per le partite ranked e unranked}
\label{indici_sintesi_winrate_sintetico}
\begin{tabular}{|r|r|r|}
\hline
\multicolumn{1}{|l|}{} & \multicolumn{1}{l|}{ranked} & \multicolumn{1}{l|}{unranked} \\ \hline
media               &  0.5008  &   0.5000 \\ \hline
mediana             &  0.5074  &  0.5132 \\ \hline
deviazione standard & 0.08350946  &   0.07758461 \\ \hline
varianza            &  0.00697383  &   0.006019371 \\ \hline
\end{tabular}
\end{table}
\begin{table}
\centering
\caption{Tabella riassuntiva degli indici simmetrici per winrate nelle partite ranked e unranked}
\label{indici_simmetrici_winrate_sintetico}
\begin{tabular}{|r|r|r|}
\hline
\multicolumn{1}{|l|}{} & \multicolumn{1}{l|}{ranked} & \multicolumn{1}{l|}{unranked} \\ \hline
skewness    &  -0.4500729  &   -0.5209041     \\ \hline
curtosi     &  3.495748  &   2.78405     \\ \hline
\end{tabular}
\end{table}
\begin{figure}[htbp]
\begin{center}
% \hspace*{-0.15\linewidth}
\includesvg[scale=0.9]{./FigureSintetico/differenze_ranked_unranked/differenza_winrate_ranked_unranked_sintetico.svg}
\caption{Nella seguente figura è riportata la differenza dei winrate per i singoli Hero nelle partite ranked e unranked}
\label{differenza_winrate_ranked_unranked_sintetico}
\end{center}
\end{figure}
In conclusione per quanto riguarda il pickrate, è presente una bassisima variabilità per il singolo Hero ma una variazione della distribuzione del pickrate. Per il winrate, le distribuzioni risultano essere simili tra di loro mentre per i singoli Hero sussistono comunque delle differenze importanti.

\subsection{RQ3:I winrate dei Radiant e dei Dire sono differenti?}
Si verifica ora se sussistono delle differenze notevoli di winrate tra i due team di Dota2. I winrate per ciascuno dei due team sono riportati all'interno della seguente tabella~\ref{winrate_team_sintetico}. Rispetto al dataset originale si può notare che la differenza di winrate tra i due team risulta essere molto più marcata. Se si dovesse interpretare l'insieme delle partite generate come un insieme di partite in un determinato stato del gioco, tale stato di gioco sarebbe da considerare estremamente sbilanciato a favore del team 1. 
\begin{table}
\centering
\caption{Tabella riassuntiva del winrate di ciascuno dei due team per le partite ranked e unranked}
\label{winrate_team_sintetico}
\begin{tabular}{|r|r|r|}
\hline
\multicolumn{1}{|l|}{} & \multicolumn{1}{l|}{ranked} & \multicolumn{1}{l|}{unranked} \\ \hline
team 1  &  57.77\%  &   55.66\% \\ \hline
team 2  &  42.23\%  &   44.34\%  \\ \hline
\end{tabular}
\end{table}

\subsection{RQ4:Esiste una correlazione tra pickrate e winrate?}
Si procede quindi a verificare se esiste una correlazione tra winrate e pickrate nei dati generati in maniera sintetica. Di seguito si mostrano i diagrammi di dispersione~\ref{plot_winrate_pickrate_ranked_raw_sintetico}~\ref{plot_winrate_pickrate_unranked_raw_sintetico} per le partite ranked e unranked. A partire da questi ultimi è possibile notare come sia difficile individuare una correlazione tra pickrate e winrate. Infatti, tutte quante le tecniche che sono state utilizzate sul dataset originale hanno restituito dei risultati per r-squared ancora più bassi(spesso quasi approssimabbili a zero). \\ 
\begin{figure}[htbp]
\begin{center}
\includesvg[scale=0.75]{./FigureSintetico/correlazione_pickrate_winrate/plot_winrate_pickrate_ranked_sintetico.svg}
\caption{Diagramma di dispersione del winrate e del pickrate degli Hero nelle partite classificate}
\label{plot_winrate_pickrate_ranked_raw_sintetico}
\includesvg[scale=0.75]{./FigureSintetico/correlazione_pickrate_winrate/plot_winrate_pickrate_unranked_sintetico.svg}
\caption{Diagramma di dispersione del winrate e del pickrate degli Hero nelle partite non classificate}
\label{plot_winrate_pickrate_unranked_raw_sintetico}
\end{center}
\end{figure}
Si conferma che anche per quanto riguarda i dati generati in maniera sintetica non sono presenti delle correlazioni banali tra pickrate e winrate. Per quanto riguarda però i dati sintetici, pur divididendo in quattro quadranti lo scatterplot non si riesce a trarre nessuna conclusione che avrebbe un significato rispetto al dominio di riferimento. Questo perchè i dati generati hanno una variabilità del pickrate praticamente inesistente, tanto da renderli palesemente artificiali. \\
La relazione tra pickrate e winrate che era difficile da descrivere ma che era presente per quanto riguarda i dati del dataset originale, sembra quindi non esser stata replicata dalla generazione dei dati. 


\newpage
\subsection{RQ5:Esistono degli attributi più incisivi all'interno del gioco?}
Si tenta ora di rispondere alla research question 5 anche per quanto riguarda il dataset sintetico. I concetti di pickrate e winrate, sono i medesimi di quando è stata effettuata l'analisi per il dataset originale. Nel seguente grafico~\ref{pickrate_attributi_ranked_sintetico} è stato riportato il pickrate per ciascuno degli attributi per le partite ranked. Rispetto al dataset originale, si può notare che i pickrate degli attributi sono molto più simili tra di loro. Per quanto riguarda invece il winrate, si osservano diverse differenze nei valori associati agli attributi Strength e Agility, che risultano nettamente superiori rispetto a Intelligence e Universal. Di conseguenza, emerge una dicrepanza rispetto al dataset originale, in cui tutti i valori di winrate sono simili (intorno al 50\%) e solo Intelligence risulta effettivamente inferiore rispetto agli altri attributi.
\begin{figure}[htbp]
\centering
\begin{multicols}{2}
\hspace*{-0.2\linewidth}
\includesvg[scale=0.6]{./FigureSintetico/attributi_winrate_pickrate/pickrate_attributi_ranked_sintetico.svg}
\caption{La seguente figura è un barplot del pickrate per ciascuno degli attributi nelle partite ranked del dataset sintetico}
\label{pickrate_attributi_ranked_sintetico}
\hspace*{-0.1\linewidth}
\includesvg[scale=0.6]{./FigureSintetico/attributi_winrate_pickrate/winrate_attributi_ranked_sintetico.svg}
\caption{ La seguente figura è un barplot del winrate per ciascuno degli attributi nelle partite ranked }
\label{winrate_attributi_ranked_sintetico}
\end{multicols}
\end{figure} 
All'interno della seguente figura~\ref{pickrate_attributi_merged_sintetico} è stato riportato un baplot per ciascuno degli attributi nelle partite ranked. All'interno di ciascun grafico è stata utilizzata una colonna per riportare la percentuale di squadre in cui compaiono esattamente un certo numero di Hero con un determinato attributo. Anche per il dataset sintetico, si è deciso di escludere tutte quante le configurazioni di team che hanno una frequenza minore del 5\% sono state tagliate dal resto dello studio per questa research question. Le frequenze relative di team in cui compaiono un esatto numero di Hero con un determinato attributo hanno valori molto simili tra di loro. Questa è una notevole differenze rispetto al dataset originale.
\begin{figure}[htbp]
\begin{center}
\includesvg[scale=0.9]{./FigureSintetico/attributi_winrate_pickrate/pickrate_total_ranked_sintetico.svg}
\caption{Nella seguente figura sono riportati 4 barplot, uno per ogni attributo del gioco. All'interno di ciascun grafico è stata utilizzata una colonna per riportare la percentuale di squadre in cui compaiono esattamente un certo numero di Hero con un determinato attributo. I seguenti grafici sono riferiti alle partite ranked del dataset sintetico. }
\label{pickrate_attributi_merged_sintetico}
\end{center}
\end{figure}
All'interno della seguente figura~\ref{pickrate_attributi_merged_sintetico} invece sono ripotati i barplot del winrate per ciascuno degli attributi. In questo caso è possibile notare, che rispetto al caso originale per gli attributi Strenght e Agility la situazione sembra essere più o meno invariata, sono invece presenti delle notevoli differenze per quanto riguarda Intelligence e Universal, per i quali la cosa migliore è avere esattamente un Universal o Intelligence in squadra.
\begin{figure}[htbp]
\begin{center}
\includesvg[scale=0.9]{./FigureSintetico/attributi_winrate_pickrate/winrate_total_ranked_sintetico.svg}
\caption{Nella seguente figura sono riportati 4 barplot, uno per ogni attributo del gioco. All'interno di ciascun grafico è riportato il winrate dei team che possiedono un certo numero di Hero con un determinato attributo. I seguenti grafici sono riferiti alle partite ranked dell'attributo sintetico.}
\label{winrate_attributi_merged_sintetico}
\end{center}
\end{figure}
Osservando i grafici è possibile quindi notare che per le partite ranked del dataset sintetico la scelta migliore è quella di avere un Hero per ciascuno degli attributi. 
Per quanto riguarda le partite unranked, dai seguenti grafici (~\ref{pickrate_attributi_unranked_sintetico}~\ref{winrate_attributi_unranked_sintetico}~\ref{pickrate_attributi_unranked_merged_sintetico}~\ref{winrate_attributi_unranked_merged_sintetico}) si nota che la situazione rimane sostanzialmente invariata rispetto alle partite ranked. L'unica differenza significativa è che, per l'attributo Agility, la scelta ottimale è avere due giocatori in squadra anziché uno. Pertanto, nel dataset sintetico, non sussiste quindi nessuna particolare differenza su quelli che sono gli attributi più convenienti tra le partite ranked e quelle unranked del dataset sintetico, a differenza del dataset originale.
\begin{figure}[htbp]
\centering
\begin{multicols}{2}
\hspace*{-0.2\linewidth}
\includesvg[scale=0.6]{./FigureSintetico/attributi_winrate_pickrate/pickrate_attributi_unranked_sintetico.svg}
\caption{La seguente figura è un barplot del pickrate per ciascuno degli attributi nelle partite unranked del dataset sintetico}
\label{pickrate_attributi_unranked_sintetico}
\hspace*{-0.1\linewidth}
\includesvg[scale=0.6]{./FigureSintetico/attributi_winrate_pickrate/winrate_attributi_unranked_sintetico.svg}
\caption{ La seguente figura è un barplot del winrate per ciascuno degli attributi nelle partite unranked del dataset sintetico }
\label{winrate_attributi_unranked_sintetico}
\end{multicols}
\end{figure} 
\begin{figure}[htbp]
\begin{center}
\includesvg[scale=0.9]{./FigureSintetico/attributi_winrate_pickrate/pickrate_total_unranked_sintetico.svg}
\caption{Nella seguente figura sono riportati 4 barplot, uno per ogni attributo del gioco. All'interno di ciascun grafico è stata utilizzata una colonna per riportare la percentuale di squadre in cui compaiono esattamente un certo numero di Hero con un determinato attributo. I seguenti grafici sono riferiti alle partite unranked del dataset sintetico. }
\label{pickrate_attributi_unranked_merged_sintetico}
\end{center}
\end{figure}
\begin{figure}[htbp]
\begin{center}
\includesvg[scale=0.9]{./FigureSintetico/attributi_winrate_pickrate/winrate_total_unranked_sintetico.svg}
\caption{Nella seguente figura sono riportati 4 barplot, uno per ogni attributo del gioco. All'interno di ciascun grafico è riportato il winrate dei team che possiedono un certo numero di Hero con un determinato attributo. I seguenti grafici sono riferiti alle partite unranked dell'attributo sintetico.}
\label{winrate_attributi_unranked_merged_sintetico}
\end{center}
\end{figure}


\newpage
\subsection{RQ6:Esistono configurazioni dei team più convenienti?}
Nella seguente research question 6 si cercherà di definire se esistono dei team che sono più convenienti dal punto di vista statistico per ottenere la vittoria. Anche in questo caso si è deciso di scartare tutte quante le configurazioni di team che sono presenti in meno del 10\% delle partite. In questo caso abbiamo un team in meno nella top rispetto alle classifiche del dataset originale. \\
In questo caso come è possibile osservare dai winrate e dalle configurazioni dei team che sono presenti all'interno delle tabelle~\ref{tabella_winrate_team_ranked_sintetico}~\ref{tabella_winrate_team_unranked_sintetico} sia nelle partite ranked che in quelle unranked i team presenti sono gli stessi anche se con delle differenti percentuali di vittoria. \\
Come nel caso del dataset originale, le configurazioni di team più vincenti sono coerenti rispetto a quelli che sono i risultati ottenuti nelle research question precedente, quindi durante l'analisi del winrate per attributi. Si era infatti concluso che convenisse sia nelle partite ranked che in quelle unranked avere almeno un Hero per ciascun attributo per massimizzare le possibilità di vittoria. \\
In modo simile al dataset originale, si può concludere che anche in questo caso esistono configurazioni di team più convenienti rispetto ad altre. L’unica differenza significativa è che, nel dataset sintetico, le configurazioni di team più vincenti presentano una percentuale di vittoria decisamente più elevata.
\begin{table}[htbp]
\centering
\caption{Nella seguente tabella sono riportati le 4 configurazioni di team che hanno il winrate più alto all'interno delle partite ranked. }
\label{tabella_winrate_team_ranked_sintetico}
\begin{tabular}{|r|r|r|r|r|}
\hline
\multicolumn{1}{|l|}{Strength} & \multicolumn{1}{l|}{Intelligence} & \multicolumn{1}{l|}{Agility} & \multicolumn{1}{l|}{Universal} & \multicolumn{1}{l|}{WinRate} \\ \hline
2 & 1 & 1 & 1 &   58  \% \\ \hline
1 & 1 & 2 & 1 &   57  \% \\ \hline
1 & 2 & 1 & 1 &   49.5\%  \\ \hline
1 & 1 & 1 & 2 &   48.4\%  \\ \hline
\end{tabular}
\end{table}
\begin{table}[htbp]
\centering
\caption{Nella seguente tabella sono riportati le 4 configurazioni di team che hanno il winrate più alto all'interno delle partite unranked.}
\label{tabella_winrate_team_unranked_sintetico}
\begin{tabular}{|r|r|r|r|r|}
\hline
\multicolumn{1}{|l|}{Strength} & \multicolumn{1}{l|}{Intelligence} & \multicolumn{1}{l|}{Agility} & \multicolumn{1}{l|}{Universal} & \multicolumn{1}{l|}{WinRate} \\ \hline
1 & 1 & 2 & 1 &   57.1 \% \\ \hline
1 & 2 & 1 & 1 &   52.7 \%  \\ \hline
2 & 1 & 1 & 1 &   50.2 \%  \\ \hline
1 & 1 & 1 & 2 &   43.5 \% \\ \hline
\end{tabular}
\end{table}

\subsection{RQ7:Qual'è il rank delle istanze del dataset?}
Infine si procede a rispondere alla research question 7 anche per il dataset sintetico, all'interno della quale si cercherà di determinare se l’elo per le partite ranked e unranked segue o meno una distribuzione nota e se questa distribuzione individuata è riconducibile dal campione alla popolazione. Come per il dataset originale il valore dell’elo di ogni partita del dataset sintetico è stato calcolato attraverso l’utilizzo della medesima formula utilizzata per il dataset originale. \\
All’interno delle seguenti figure () sono presenti i grafici delle distribuzioni dell'elo, rispettivamente per le partite ranked e unranked. Proprio come nel caso del dataset originale entrambi i grafici ci suggeriscono che l’elo abbia una distribuzione normale. Quindi per determinare ciò abbiamo utilizzato il test del chi-quadro per verificare se la distribuzione è effettivamente una normale o meno.
Nel caso delle partite ranked il valore di chi-quadro è 15.16329 mentre per le partite unranked è 9.958115, dato che in entrambi i casi il valore non rientra nell’intervallo di accettazione del test del chi-quadro(0.05063562, 7.377759) si può affermare che l’elo sia per le partite ranked che per quelle unranked non segue una distribuzione normale. \\
\begin{figure}[htbp]
\centering
\begin{multicols}{2}
\hspace*{-0.1\linewidth}
\includesvg[scale=0.7]{./FigureSintetico/distribuzione_elo_ranked_sintetico.svg}
\caption{Nella seguente figura è presente un grafico della distribuzione dell'elo calcolato all'interno delle partite ranked}
\label{distribuzione_elo_ranked_sintetico}
\hspace*{-0.1\linewidth}
\includesvg[scale=0.7]{./FigureSintetico/distribuzione_elo_unranked_sintetico.svg}
\caption{Nella seguente figura è presente un grafico della distribuzione dell'elo calcolato all'interno delle partite unranked}
\label{distribuzione_elo_unranked_sintetico}
\end{multicols}
\end{figure} 
Di conseguenza, anche se la forma della distribuzione non sembra seguire una distribuzione di Poisson, si è comunque deciso di applicare il test del chi-quadro per verificarlo. Nel caso delle partite ranked, il valore del chi-quadro è 397.6751, mentre per le partite unranked è 879.4748. Poiché questi valori non rientrano minimamente nell’intervallo di accettazione del test, si può confermare che le osservazioni fatte sulle distribuzioni dell’Elo sono corrette. \\
Si conclude quindi che entrambe le distribuzioni non seguono alcuna distribuzione nota, il che impedisce di proseguire lo studio e di determinare il valore medio dell’Elo della popolazione per il dataset sintetico. 


\subsection{Conclusioni}
In conclusione, la generazione di dati sintetici è riuscita a simulare le differenze tra le partite ranked e unranked per quanto riguarda il pickrate e il winrate degli Hero. Per il pickrate però è necessario notare che rispetto al dataset originale i dati generati dal dataset sintetico dimostrano una perfetta ripartizione del pickrate tra i vari Hero. % cioè tutti gli Hero vengono pikkati quasi allo stesso modo
La generazione dei dati non è riuscita a cogliere le complicate relazioni che intercorrono tra i vari elementi che costituiscono le partite. Questo è dimostrato dal fatto che non è stato possibile neanche individuare delle aree all'interno dello scatterplot winrate/pickrate, come invece è stato possibile per il dataset originale. I dati generati variano sensibilmente anche per quanto riguarda il winrate dei team Radian e Dire. 
Se per quanto riguarda il dataset originale la distribuzione dell'elo per le partite ranked è riconducibile a una distribuzione nota(normale) questo non è vero per il dataset sintetico. \\
Il dataset sintetico generato in questo caso non si ritiene attendibile per riuscire a migliorare l'analisi statistica oppure per migliorare i risultati della regressione. A nostro parere, le feature del dataset risultano inadeguate per cercare di generare delle nuove istanze di partite che possano rispecchiare le relazioni tra intercorrono tra gli elementi del gioco e quindi risultare utili per migliorare i risultati dell'analisi statistica.