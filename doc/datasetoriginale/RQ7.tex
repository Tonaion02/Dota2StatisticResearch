In Dota2 vincendo le partite classificate è possibile accumulare punti rank. Questi punti determinano l'elo(praticamente il rango) del giocatore che viene poi utilizzato durante la creazione dei due team per una determinata partita(matchmaking). \\
I team per le partite vengono creati con strategie differenti rispetto al fatto che le partite siano ranked o unranked. Per le partite unranked non si utilizza l'elo per determinare da quali giocatori saranno composti i team. Questa informazione viene invece utilizzata per le partite ranked.  \\
Se i rank infatti sono diventati nel tempo un motivo di orgoglio e degli obbiettivi da raggiungere per i giocatori, originariamente erano stati sopratutto pensati per essere utilizzati come parametro ai fini di organizzare le partite in maniera che potesse esserci un agonismo sensato. \\ 
L'obbiettivo di questa research question è provare a determinare attraverso altri parametri il rank delle partite, verificare se la distribuzione del rank delle partite è nota e se questa distribuzione individuata è riconducibile dal campione analizzato alla popolazione. Anche in questo caso è stato fondamentale analizzare i due campioni(ranked e unranked) in maniera distinta. \\
All'interno del dataset non è presente come informazione il rank di ciascuno dei giocatori. Di conseguenza, determinare il rank di una partita non può essere fatto analizzando quei valori. Al variare dell'elo dei partecipanti ad una partita possono variare alcuni aspetti, che diventano quindi indici che ci possono suggerire qual'è il rank della partita. In particolare ciascuno degli Hero nel gioco possiede un parametro di complessità che va da 1 a 3, che indica quanto l'Hero è difficile da utilizzare. E' plausibile che l'utilizzo di Hero più complessi sia maggiormente diffuso in partite dove ci sono giocatori con un elo maggiore. \\
Un altro parametro influenzato dal rank, riguarda il numero di counter-pick che vengono effettuati. Un counter-pick non è altro che una scelta di un personaggio fatta per mettere in difficoltà l'avversario in base alla sua scelta. All'interno di Dota2 per ciascun Hero è possibile individuare almeno un altro Hero che attraverso le sue caratteristiche nullifica le potenzialità dell'altro. E' logico pensare che un numero maggiore di counter-pick saranno presenti all'interno di partite dove ci sono giocatori con un elo maggiore, che conoscono meglio il gioco e sanno quindi giocare meglio di strategia. \\
Dopo aver considerato l'elo come una variabile aleatoria sono stati utilizzati i valori di complessità e i counter-pick per determinarne il valore. In particolare, la complessità della partità è stata definita come la somma delle complessità degli Hero che compaiono in una partita, mentre il valore di counter-pick è il numero di counter-pick che sono stati effettuati per quella partita. Il valore del rank sarà quindi dato la seguente combinazione lineare dei valori di counter-pick e complessità  \( 500 + \left( \left( 0.85 \cdot \frac{C - 10}{20} + 0.15 \cdot \frac{CP}{25} \right) \cdot 5000 \right) \). All'interno della seguente formula è possibile osservare due parametri C e CP, che rappresentano rispettivamente la complessità e il counter-pick della partita. Il valore C viene normalizzato sottraendo 10 e dividendo il risultato per 20, in quanto il valore della complessità nella partita può variare tra 10 e 30. Mentre invece il valore di CP viene diviso per 25,in quanto il numero massimo di counter-pick in una partita è 25. Il valore massimo è dato dal fatto che attraverso un pick è possibile al più fare un counter-pick per i cinque personaggi della squadra avversaria. Per quanto riguarda i pesi che sono stati scelti per il parametro C e il parametro CP ci consentono di valutare principalmente il valore della complessità, che riteniamo più incisiva dei counter-pick. Dato che il valore dell'elo nel gioco varia tra 500 e 5500, si è deciso di passare i valori in questa scala. \\
\begin{figure}[htbp]
\centering
\begin{multicols}{2}
\hspace*{-0.12\linewidth}
\includesvg[scale=0.7]{./Figure/analisi_elo/distribuzione_elo_ranked.svg}
\caption{Nella seguente figura è presente un grafico della distribuzione dell'elo calcolato all'interno delle partite ranked}
\label{distribuzione_elo_ranked}
\hspace*{-0.1\linewidth}
\includesvg[scale=0.7]{./Figure/analisi_elo/distribuzione_elo_unranked.svg}
\caption{Nella seguente figura è presente un grafico della distribuzione dell'elo calcolato all'interno delle partite unranked}
\label{distribuzione_elo_unranked}
\end{multicols}
\end{figure}
All'interno delle seguenti figure~\ref{distribuzione_elo_ranked}~\ref{distribuzione_elo_unranked} sono presenti dei grafici per la distribuzione dell'elo da noi calcolato rispettivamente per le partite ranked e quelle unranked. Entrambi i grafici ci suggeriscono che l'elo abbia una distribuzione normale. Successivamente verrà utilizzato il test del chi-quadro per verificare se la distribuzione è effettivamente una normale o meno.  \\
Nel caso delle partite ranked il valore di chi-quadro è 1.597444 e l'intervallo (0.05063562, 7.377759), di conseguenza il test del chi-quadro è passato. Il test del chi-quadro invece nel caso delle partite unranked ha restituito per chi-quadro il valore 13.4207 e come intervallo (0.05063562, 7.377759). Nel caso delle partite unranked il test del chi-quadro è quindi risultato non valido, di conseguenza il resto dello studio verrà portato avanti solamente per le partite ranked. \\
Sapendo che l'elo nelle partite ranked segue una distribuzione normale, si cerca ora di determinare se le considerazioni effettuate per questo campione possono essere estese alla popolazione. Nello specifico ci interessa farlo per il valore medio dell'elo della popolazione, cioè il valore medio dell'elo delle partite di Dota2. Per effettuare la stima puntuale si è utilizzato il metodo della massima verosimiglianza. In questo caso lo stimatore di massima verosomiglianza è la media campionaria dell'elo. Il valore della media campionaria dell'elo è: 1742.19, ma è scontato in questo caso che tale valore rientra nell'intervallo di confidenza(1727.62760612041 ,  1756.75258557288). E' stato così ottenuto con un grado di affidabilità del 95\% che il valore medio della popolazione rientra all'interno dell'intervallo di confidenza. \\
Per ottenere una maggiore certezza, si è deciso anche di effettuare la verifica delle ipotesi con un grado di affidabilità del 99\%. Dove nello specifico ciò che è stato fatto è un test bilaterale dove l'ipotesi nulla è che il valore medio dell'elo della popolazione è uguale a quello del nostro campione. In particolare si vuole verificare che l'ipotesi sul valore medio è vera nel caso in cui la varianza della popolazione non è nota. Dato che il test ha dato successo sull'interavallo (-1.960722, 1.960722) con il valore 0 e che il p-value(1) è magggiore di alfa(0.01) non si può rifiutare l'ipotesi nulla con un grado di significabilità del 99\%. \\

